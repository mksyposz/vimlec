\documentclass{beamer}
\usetheme{Copenhagen}
\usecolortheme{beaver}
\usepackage{tikz}
%\usepackage{hyperref}
\usepackage{url}

\title{Vim lectures}
\subtitle{The beginning}
\author{Michał Syposz}
\date{\today}

\begin{document}
    \begin{frame}
    \titlepage
    \end{frame}
    \section{Introduction}
    \begin{frame}
    \frametitle{My idea behind the lectures}
    \begin{itemize}
        \item Not all lectures will have corresponding presentations.
        \item I will try to write some sort of note to every lecture.
        \item Lectures are not going to be long. I don't want to show you 
            every command and aspect of Vim I know since there's no way you're going
            to remember and use all of it after one meeting.
        \item I want to encourage terminal usage, I will put most of the lecture notes
            on github.\footnote{If someone doesn't remember how to properly use github 
            it's good opportunity for them. I can send some good materials explaining
            the usage of git}
    \end{itemize}
    \end{frame}
    \subsection{Vim or NeoVim?}
    \begin{frame}
    \frametitle{Vim vs NeoVim}
        So let's get one thing straight. When we're talking about Vim we don't really 
        mean Vim. We mean NeoVim.\\
        Basically Vim creator was close-minded on many features he could add to it.
        Some developers decided to fork Vim and write extensions themselves. That's how
        NeoVim was created. It was much faster than normal Vim because for example
        it was asynchronous. After Vim creator found out about it 
        he \textbf{copied all the features he was refusing to write for years}. As smart 
        people we don't want to support anyone like that. Therefore we use NeoVim instead
        of Vim.\\
        \textit{Quick disclaimer}: everything I'm going to show you works in Vim as well. 
    \end{frame}
    \subsection{Motivation}
    \begin{frame}
    \frametitle{Motivation behind learning Vim} 
    \begin{enumerate}
        \item Programming is not really about writing code, it's mostly about modyfing it.
            Vim possibilities makes that super fast.
        \item Vim is fast AF. I think like only nano is faster to open big files.
        \item It's almost everywhere. If I SSHed into a server and there would be no
            Vi/Vim/NeoVim installed I would be surprised.
        \item You don't need to use mouse anymore. Using mouse a lot is bad for your back 
            apparently.
        \item It makes programming smooth, fast and exciting. You'll see.
    \end{enumerate}
    \end{frame}
    \section{The start}
    \subsection{Modes}
    \begin{frame}
    \frametitle{Modes}
    Vim has modal editing. That means:\\
    Vim has 4 real modes: Normal, Insert, Visual and Replace.\\
    There's a fifth called command line mode.
    \begin{center}
        \begin{tikzpicture}
        \node (C) at (-2,4) {CLi};
        \node (V) at (0,0) {Visual};
        \node (R) at (4,0) {Replace};
        \node (I) at (4,4) {Insert};
        \node (N) at (0,4) {Normal};
        \draw[<->] (N) -- (I);
        \draw[<->] (N) -- (R);
        \draw[<->] (N) -- (V);
        \draw[<->] (N) -- (C);
        \end{tikzpicture}
    \end{center}
    \end{frame}
    \begin{frame}
    \frametitle{Modes}
    The modes' names are self explanatory.
    \begin{itemize}
        \item Normal - this mode is the reason why Vim is so awesome. You can use almost
            all commands that come with Vim in it.
        \item Insert - it's for inserting text into the file.
        \item Replace - it's like insert but it's replacing the text you're writing on.
        \item Visual - highlighting
        \item Command line - you can launch Vim commands from it. Even bash or ZSH
            commands!
    \end{itemize}
    \end{frame}
    \begin{frame}
    \frametitle{Modes}
        There's lots of ways you can change modes. I will show you some easy ones for now.\\
        First of all to change any mode to Normal you press \texttt{Escape}. Since you 
        use it a lot it might be a good idea to remap it.
        \footnote{Clue: probably as me you don't use CapsLock a lot.}\\
        As you can see changing modes is only possible from Normal mode. These are 
        \textbf{basic} commands to achieve that:\\
        Entering insert mode: \texttt{i}\\
        Entering replace mode: \texttt{r,R}\\
        Entering visual mode: \texttt{v,V,ctrl+v}\\
        Entering command line mode: \texttt{:}
    \end{frame}
    \subsection{Movements}
    \begin{frame}
    \frametitle{Fundamental Movements}
        Better arrow keys in vim\\
        \begin{center}
            $\leftarrow$ \texttt{h}\\
            $\downarrow$ \texttt{j}\\
            $\uparrow$ \texttt{k}\\
            $\rightarrow$ \texttt{l}
        \end{center}
        If you use touch typing 
        \footnote{I can provide some tutorial link's for that as well}.
        As you probably as self respecting programmer should
        those letters are under your right hand.
    \end{frame}
    \begin{frame}
    \frametitle{Fundamental Movements}
        Moving over by h/l might be really annoying. Luckily there are easy commands to
        save us.\\
        \texttt{w} - jump to the beginning of next word\\
        \texttt{e} - jump to the end of next word (not used every often)\\
        \texttt{b} - jump to the beginning of previous word\\
    \end{frame}
    \subsection{Basic editing commands}
    \begin{frame}
    \frametitle{Basic editing commands}
        \texttt{y} - stands for yank $\Leftrightarrow$ copy\\
        \texttt{yy} - copies current line to Default register
        \footnote{I'm planing to discuss registers later in the lectures}\\
        \texttt{d} - stands for delete. Deleting some texts puts it in the default
        register as well\\
        \texttt{dd} - deletes current line\\
        \texttt{p} - pastes what's in Default register\\
        \texttt{u} - undoes
        When you want to edit something more than a line you can use visual mode. In
        visual mode movement and editing commands work as in normal.
    \end{frame}
    \begin{frame}
    \frametitle{Combining what we know}
    Now I'll quickly demonstrate what you can do with commands I've already showed you.
    Visual + movement + editing commands.
    \end{frame}
    \begin{frame}
    \frametitle{Most important command you'll learn this lecture}
    You save and exit files in Vim from command line mode. 
    All you need to write is:\\
    \texttt{:w} to save the file\\
    \texttt{:q} to exit Vim\\
    You can combine these commands to save file and exit vim \texttt{:wq} \\in short 
    \texttt{:x}. Exiting Vim is not always what you want to do. I'll talk about some 
    alternatives in later lectures.
    \end{frame}
    \section{TBC}
    \begin{frame}
    I think these commands are enough to start your adventure with vim.\\
    For now it's probably best if you just use a plug in for your favorite editor, 
    since we didn't modify vim at all. We'll get to the point of using vim as your 
    primary editing tool in couple of meetings. If you feel like learning more vim 
    by yourself I highly recommend ThePrimeagen on YouTube. He recently made a vim 
    intro series, which the beginning of these lectures is based upon.
    \end{frame}
    \begin{frame}
        \center
        See you in the next one
    \end{frame}
\end{document}
